\documentclass[letterpaper,11pt]{article}
\usepackage[round]{natbib}
\usepackage[margin=1in,centering]{geometry}
\usepackage{fancyhdr}
\usepackage{amsmath}
\usepackage{amssymb}
\usepackage{graphicx}
\usepackage[pdftex]{hyperref}
\hypersetup{
    pdftitle={On the Rattleback},
    pdfauthor={Dale Lukas Peterson},
    pdfsubject={dynamics and stability of the rattleback},
    pdfkeywords={rattleback, nonholonomic constraints, Kane's method}}

\pagestyle{fancy}
\fancyhead[L]{On the Rattleback, Dale L. Peterson}
\fancyhead[R]{\thepage}  % page number on the right
\fancyfoot[L,C,R]{}  %  No footer on left, center or right, on even or odd pages

\begin{document}
\abstract{This report briefly reviews previous studies of the solid body
  commonly known as a rattleback, celt, or wobblestone.  These solids, whose
  geometric and principal inertia axes are unaligned , exhibit two unexpected
  behaviors: spin reversal, and ``spin bias''.  Spin reversal occurs when the
  body is initial spun about a vertical axis in one direction and after some
  time begins to spin in the opposite direction.  ``Spin bias'' is the behavior
  observed in some rattlebacks which reverse when spun in one direction but not
  the other.  To facilitate motion visualization, comparison with and
  verification of previous work, I derive motion equations for both the
  slipping and non-slipping cases under the assumption of a quadric surface
  which can represent both ellipsoid and paraboloid surfaces, and which permits
  both energetically conservative or dissipative assumptions.  The stability of
  the equilibrium configuration is examined as a function of the spin rate.  For
  dissipation free models, linear stability analysis can only show where spin
  reversal cannot occur, not whether it will occur and why.}

  \section{Model description, outline of motion equations}
  \section{No-slip model}
  \subsection{Energetically conservative case}
  \subsection{Dissipative case}
  \section{Slipping model}
\end{document}
